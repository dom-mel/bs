\documentclass[a4paper]{article}
\usepackage[utf8]{inputenc}
\usepackage{fullpage}
\usepackage{csquotes}
\usepackage[ngerman]{babel}
\usepackage{float}
\usepackage{graphicx}
\usepackage{epstopdf}
\usepackage{subfigure}
\setcounter{secnumdepth}{-1} 
\usepackage{hyperref}
\title{Betriebssysteme Vertiefung \\ Übungskomplex 1 Teil 2}

\author{Dominik Eckelmann, Matr.-Nr.: 785856}
\date{\today}
\begin{document}

\maketitle

\tableofcontents

\section{Einleitung}
Als Teil der Lehrveranstaltung \textit{Betriebssysteme Vertiefung} im Wintersemester 2011/2012 an der \textit{Beuth Hochschule für Technik Berlin} sollte im Rahmen einer Übung mithilfe der Programmiersprache C ein Programm geschrieben werden, welches paralleles Sortieren unter Verwendung
des Quicksort Algorithmus ermöglicht.

\newpage

\section{Aufgabe 1 und 2}

Das Programm wurde mithilf der pthread Bibliothek Multithreadingfähig gemacht.
Die Zeitmessung mit `time' ergab:

\begin{description}
\item[Mit 1 Thread] xx Sekunden
\item[Mit 2 Threads] xx Sekunden
\end{description}

Die Zeitmessung direkt im Programmcode wurde mithilfe der `gettimeofday' Funktion vorgenommen.
Gemessen wird die Zeit zwischen dem starten der Threads und dem beenden der Threads.
Die Generierung, sowie die Ausgabe des Arrays wird nicht mit berücksichtigt.

Die Messungen ergaben:
\begin{description}
\item[Mit 1 Thread] xx Sekunden
\item[Mit 2 Threads] xx Sekunden
\end{description}

\section{Aufgabe 3}

In Aufgabe 3 sollen verschiedene Ausgaben Interpretiert werden.

Die Ausgabe des ersten `ps' Aufrufes ist:
\begin{verbatim}
UID        PID  PPID  C STIME TTY          TIME CMD
dominik   8433  8429  0 16:42 pts/0    00:00:00 bash
dominik   8537  8433  0 16:52 pts/0    00:00:00   vi xxx
dominik   8538  8537  3 16:52 pts/0    00:00:00     bash
dominik   8563  8538  0 16:52 pts/0    00:00:00       ps -Hf
\end{verbatim}

Die Abhängigkeiten der gezeigten Prozesse ist:
\begin{description}
\item[ps -Hf (PID: 8563)] ist der Kindprozess von `bash'
\item[bash (PID: 8538)] ist der Kindprozess von `vi xxx'
\item[vi xxx (PID: 8537)] ist der Kindprozess von `bash'
\item[bash (PID: 8433)] ist der Kindprozess vom Prozess mit der PID 8429
\end{description}

Andersrum gesagt, es wurde auf der `bash' der Befehl `vi xxx' gestartet.
Aus dem `vi' heraus wurde eine weitere `bash' gestartet in welcher der `ps'-Befehl
die Prozessausgabe von oben anzeigt.

\paragraph{}
Die Ausgabe des Befehles `ps -ef | grep -v grep | grep 8433'
\begin{verbatim}
UID       PID   PPID  C STIME TTY      TIME     CMD
dominik   8433  8429  0 16:42 pts/0    00:00:00 bash
dominik   9275  8433  0 17:09 pts/0    00:00:00 ps -ef
\end{verbatim}
Zunächst sehen wir hier wieder einen Elternprozess und sinen Kindprozess.	

\begin{description}
\item[UID] ist die Benutzer ID bzw. der Anmeldename des Benutzers, unter welchem der Prozess läuft.
\item[PID] ist die Prozess ID. Sie ist eindeutig für jeden Thread.
\item[PPID] ist die Prozess ID des Elternprozesses.
\item[CMD] ist das Programm, welches ausgeführt wird.
\end{description}

\paragraph{}
Das Kommando `exec date' ersetzt den laufenden Prozess mit dem des `date' Befehles.
Das heißt es wird das aktuelle Datum ausgegeben und danach das Programm beendet, sprich
die Shell schließt sich. Hierzu auch aus den Manpages: 
`The  exec() family of functions replaces the current process image with a new process image.'

\section{Aufgabe 4}
Um die Shellskripte zu starten müssen diese zunächst via `chmod +x script\_datei.sh'
das Recht bekommen ausgeführt zu werden.

Die Ausgabe der `ps' Kommandos zeigt den aktuellen Prozessbaum. Beim Ausführen
der `master.sh' werden jedoch die Kindprozesse child1 und child2 nicht angezeigt.
Bei den `ps' Ausgaben der `child1.sh' und der `child2.sh' werden hier jeweils
beide Kindprozesse der `master.sh' angezeigt.

Die Reihenfolge der Ausgaben ist wie folgt: Zuerst wird der Inhalt der `master.sh'
ausgegeben. Die Ausgaben der `child1.sh' und `child2.sh' finden gleichzeitig statt
und überschneiden sich.

Damit die Ausgabe `Master done' am ende erscheint muss die `master.sh' folgendermaßen
angepasst werden:

\begin{verbatim}
#!/bin/bash
####################################################
# master programm
#
####################################################
date
echo "start two child processes!"
./child1.sh &
echo "Process id from first child process $!"
./child2.sh &
echo "Process id from second child process $!"
ps -HF
wait
echo "Master done"
exit 0
\end{verbatim}

\section{Aufgabe 5}

Wenn das Kommando `:()\{ :\textbar:\& \};:' auf der Shell ausgeführt wird,
wird zunächst eine Funktion mit dem namen `:' erzeugt.
Die Funktion startet sich nun selbst (rekrusiv) in einem neuen Prozess aus. Im aktuellen
Prozess ruft sich sich auch rekrusiv aus.
Theoretisch werden nun unendlich viele Prozesse gestartet.
In der Praxis legt dieses Kommando den ausführenden
Computer lahm. Diese Art von Programmen werden auch forkbomb genannt.

Das C-Programm stellt die gleiche Funktionalität bereit. Der Unterschied ist lediglich das
es iterativ arbeitet und so in einer endlosschleife neue Prozesse startet.

\end{document}
