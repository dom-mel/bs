\documentclass[12pt,a4paper]{article}
\usepackage[utf8x]{inputenc}
\usepackage{ucs}
\usepackage{makeidx}
\setcounter{secnumdepth}{-1} 
\author{Dominik Eckelmann}
\title{Betriebssysteme Vertiefung\\Übung 2}
\date{}
\begin{document}
\maketitle

\section{Einleitung}
Als Teil der Lehrveranstaltung \textit{Betriebssysteme Vertiefung} im Wintersemester 2011/2012 an der \textit{Beuth Hochschule für Technik Berlin} sollte im Rahmen einer Übung ein C-Programm zu
entwickeln, welches Zufallszahlen erzeugt und diese sortiert ausgibt. Es soll festgestellt
werden, wie viele Zahlen auf diese Weise sortiert werden können.

Zudem sollen drei Unterschiedliche Programme, welche PI berechnen, ausgeführt und ihre Laufzeit
gemessen werden.

\section{Umgebung}
Das Testsystem ist ein Intel(R) Core(TM)2 Duo L7500 @ 1.60GHz CPU, mit 4GB RAM von denen aber nur 2,9GB adressierbar sind.
Das Betriebssystem ist ein Ubuntu 10.10 (Maverick) - 32Bit mit 2.6.35-30-generic Kernel.

\section{Zufallszahlen sortieren}
Die Anzahl der zu sortierenden Zufallszahlen kann in der sortieren.h festgelegt werden.
Auf dem Testsystem kann das Betriebssystem maximal 547.483.646 Werte sortieren.
Diese Grenze zeigt sich am Arbeitsspeicher, der so voll ausgelasstet ist.

\section{Zeitmessungen}

Tabelle \ref{zeitmessung} zeigt die gemessenen Ergebnisse.
Bei pi\_v40 und pi\_v41 entsprechen sie den Erwartungen.
Wenn diese in einem Thread ausgeführt werden haben pi\_v40 und pi\_v41
ungefähr die gleiche Laufzeit. Wenn man pi\_v41 mit 2 Threads ausführt
erreicht man eine Verbesserung der Laufzeit etwas unterhalb der 50%.
Bei 4 Threads verbessert sich die Laufzeit nicht weiter. Dies is insofern
ein zu erwartendes Ergeniss, da das Testsystem nur 2 Cores besitzt auf denen
es diese Ausführen kann.

Im Falle von pi\_v42 kann keine Aussage getroffen werden, da das Programm PI nicht korrekt annähert.
Das Ergebnis ist `nan'.

\begin{table}
\begin{tabular}{|l|r|}
\hline pi\_v40 & 21,403s \\ 
\hline pi\_v41 (1 Thread) & 21,739s \\ 
\hline pi\_v41 (2 Thread) & 12,611s \\ 
\hline pi\_v41 (4 Thread) & 13,759s \\ 
\hline pi\_v42 & 1min 23,915s - Fehlerhaftes ende \\ 
\hline 
\end{tabular} 
\caption{Ergebnisse der Zeitmessungen aus Übung 2}
\label{zeitmessung}
\end{table}
\end{document}